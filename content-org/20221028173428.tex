% Created 2022-10-30 日 22:57
% Intended LaTeX compiler: pdflatex
\documentclass[11pt]{article}
\usepackage[utf8]{inputenc}
\usepackage[T1]{fontenc}
\usepackage{graphicx}
\usepackage{longtable}
\usepackage{wrapfig}
\usepackage{rotating}
\usepackage[normalem]{ulem}
\usepackage{amsmath}
\usepackage{amssymb}
\usepackage{capt-of}
\usepackage{hyperref}
\author{Jun Gao}
\date{\textit{[2022-10-28 五 17:34]}}
\title{basic concept}
\hypersetup{
 pdfauthor={Jun Gao},
 pdftitle={basic concept},
 pdfkeywords={},
 pdfsubject={},
 pdfcreator={Emacs 28.2 (Org mode 9.5.5)}, 
 pdflang={English}}
\begin{document}

\maketitle
\tableofcontents

\href{20221028173756.org}{representation learning}

\href{20221028174331.org}{normalization}

\section{QR分解}
\label{sec:orgceb4e2f}
将矩阵分解为\href{20221028175310.org}{正交矩阵}Q和一个上三角矩阵的乘积形式

\section{LU分解}
\label{sec:org313a44f}
将矩阵分解为一个下三角矩阵与上三角矩阵的乘积形式

\section{\href{20221028175536.org}{manifold流形}}
\label{sec:orgfa2ae31}

\section{softmax}
\label{sec:orgd8707d9}

\section{sigmoid}
\label{sec:org6240df0}

\section{相对熵}
\label{sec:org4f90000}
KL散度,表示同一随机变量的不同分布之间的距离

\section{交叉熵}
\label{sec:org4eef4ea}
描述用分布 \(q(x)\)
$q(x)$
\end{document}